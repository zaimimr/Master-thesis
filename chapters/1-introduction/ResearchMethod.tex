
\begin{comment}
While theory and analysis can be used for heuristics and rationale when designing models,
modern deep learning is commonly researched through experiments where multiple
alternatives are compared based on relevant metrics.
This thesis employs an experimental research method to study the problem area and
answer the research questions. Generated models from the project experiments are
compared based on pre-specified evaluation criteria on dedicated training and validation
sets to advance the models iteratively. Lastly, the final models are compared and evaluated
on an unseen test set.
\end{comment}

\begin{comment}
    This project is based on a quantitative study where alternative solutions are eval-
uated based on their statistical performance on the relevant problem. Initially,relevant approaches are assessed to gather information about the state-of-the-art methods related to the problem. Based on the analysis of these methods, we hy-
pothesize what can be improved with the given methods and use this as a guideline when proposing our method. Appropriate evaluation metrics are selected before evaluating the proposed method and comparing it with the state-of-the-art approaches, and a dataset on which experiments are performed is acquired. Subsequently, experiments are carried out, and comparisons of the different methods can be obtained. Accordingly, observations and conclusions are extracted from the
experiments.
\end{comment}
\section{Research Method}
The researchers conducted a comprehensive literature review in the fall of 2022 to understand current state-of-the-art approaches in the field of \gls{NAS}. In addition, a systematic review of relevant studies focused on performance predictors within the \gls{NAS} domain was also carried out. The project's conclusion presents several recommendations for advancing research in the \gls{NAS} field based on the insights gained from the review.

The research plan involves conducting multiple quantitative investigations to examine the performance and behaviour of zero-cost proxies on a \gls{HAR} dataset. Subsequently, the study aims to explore the potential of these proxies in improving \gls{NAS} processes.


