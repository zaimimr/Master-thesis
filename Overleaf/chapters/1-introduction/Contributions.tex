\section{Contributions}

This thesis presents several critical contributions to the \gls{NAS} field with \glspl{GCN} for \gls{HAR}. The following are the primary contributions made by this study:

\textbf{Investigation of Zero-Cost Proxies:} In this thesis, a thorough analysis of the relationship between zero-cost proxies and the performance of \gls{GCN} models in \gls{HAR} tasks is conducted. The study sheds light on the usefulness of using zero-cost proxies to estimate the performance of \gls{GCN} architectures without costly training. 

\textbf{Combining Zero-Cost Proxies:} The thesis presents methods, such as the majority vote method and the weighted arithmetic mean method, for combining different zero-cost proxies to improve the efficiency of the \gls{NAS} algorithms. 

\textbf{Environmental Considerations:} The study highlights the importance of considering the environmental implications of \gls{NAS} and artificial intelligence research. Furthermore, the study contributes to developing more sustainable practices by reducing the \gls{NAS} process's computational demands and training time.

In view of these contributions, the work in this thesis advances the understanding of \gls{NAS} with \glspl{GCN} in the context of \gls{HAR} and provide a foundation for further exploration of zero-cost proxies and their potential applications.