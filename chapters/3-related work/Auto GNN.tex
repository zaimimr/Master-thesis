\subsection{Auto-GNN}

\Gls{Auto-GNN} is a proposed method for automatically finding an optimal architecture of graph neural networks for a given graph-based task presented by \cite{zhou2019auto}. It is one of the earlier papers within GCN-NAS. The technique uses a reinforcement learning-based approach to search for an architecture that maximises task performance. Auto-GNN uses an RL agent to learn how to select and combine various GCN layers to construct an optimal GNN architecture. 

The RL agent compiles the candidate architecture before the architecture can be trained on a graph classification task. During the training, the agent uses the model's accuracy on the validation set as the reward signal. As a result, the process is highly time-consuming as it requires training the network for multiple iterations of training and evaluations. 