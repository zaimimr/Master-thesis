\chapter*{Sammendrag}

I det raskt utviklende feltet Graph Convolutional Networks (GCNs), forblir valgprosessen av arkitektur gjennom Neural Architecture Search (NAS) utfordrende på grunn av dens høye tekniske krav. Hovedfokuset med denne avhandlingen er å undersøke bruken og ytelsen av zero-cost proxies for å evaluere GCN innenfor konteksten av oppgaver relatert til gjenkjenning av menneskelig aktivitet (HAR) som et første steg mot å bruke dem i en NAS-algoritme. Basert på behovet for mer forskning i feltet, tar studien sikte på å bygge bro over dette gapet ved å evaluere forskjellige zero-cost proxies på GCN-arkitekturer. 

Så vidt vi vet, er studien den første i litteraturen til å utforske hvordan zero-cost proxies presterer på GCN. Vi gjorde en kvantitativ studie på hvordan zero-cost proxies presterer ved å lage en omfattende benchmark av trente arkitekturer for HAR-oppgaver.

Gjennom en serie analyser og eksperimenter, viste studien at integrering av zero-cost proxies kan betydelig forbedre effektiviteten og nøyaktigheten til NAS-algoritmer. Resultatene viser at de best presterende zero-cost proxyene viste en Spearman Rank Correlation ($\rho$) på omtrent $0.8$, noe som indikerer en veldig sterk korrelasjon. Imidlertid ble ingen betydelig forbedring i korrelasjon oppdaget når arkitekturene ble analysert etter å ha blitt trent i flere epoker, noe som antyder at zero-cost proxies er mest effektive ved initialiseringen av det nevrale nettverket. Forsøk på å kombinere zero-cost proxies ved hjelp av stemmegiving og vektet aritmetisk gjennomsnitt viste potensial, men ga ikke noen betydelig forbedring sammenlignet med å bruke zero-cost proxies individuelt. 